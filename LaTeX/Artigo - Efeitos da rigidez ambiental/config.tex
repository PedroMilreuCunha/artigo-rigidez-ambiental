%\documentclass[12pt, openright,oneside, a4paper, english, brazil]{abntex2}
\documentclass[12pt,
               openright,
               oneside,
               a4paper,
							 section=TITLE,     % COMO SÃO AS LETRAS MAIÚSCULAS EM SEÇÃO
               subsection=Title,  % EM DIANTE ESCRITO NORMALMENTE
               english,brazil]{article}

\usepackage[a4paper,left=2.5cm,right=2.5cm,top=2.5cm,bottom=2.5cm]{geometry}

% ----------------------------
% Pacotes básicos 
% ----------------------------
% Referências
\usepackage[brazilian,hyperpageref]{}	
\usepackage{hyperref}
\usepackage[alf]{abntex2cite}			

% Fonte e codificação (acentuação)
%\usepackage{lmodern}       
%\renewcommand{\sfdefault}{ppl}
%\usepackage[T1]{fontenc}	
\usepackage{fontspec}
\usepackage{times}
\usepackage[T1]{fontenc}
\renewcommand*\familydefault{\sfdefault}
% Tabelas e Figuras
\usepackage{ctable}             
\usepackage{multirow}
\usepackage{array}
\usepackage{float}              
\usepackage{longtable}          
\usepackage{xcolor,colortbl}    
\usepackage{booktabs}           
\usepackage{graphicx}			
\usepackage{caption}  
\usepackage{subcaption}
\usepackage{epstopdf}    
\usepackage{rotating}

% Equações e símbolos
\usepackage{amsmath}            
\usepackage{nomencl} 
\usepackage{cleveref} 
\usepackage{calrsfs}
\usepackage{amssymb}
\usepackage{psfrag}
\usepackage[brazil]{babel}
% Texto
%\usepackage[showframe,heightrounded]{geometry}
\usepackage{geometry}
\usepackage{indentfirst}		
\usepackage{color}				
\usepackage{microtype} 			
\usepackage{lastpage}			
\usepackage{enumitem}                % Para os itens em letras
\usepackage{lipsum}
\usepackage{lastpage}
\usepackage{tablefootnote}
\usepackage{pdflscape}
\usepackage{scalefnt}
\usepackage[stable]{footmisc}
\usepackage{ragged2e}
% ----------------------------------------------------------
% Configurações do texto
% ----------------------------------------------------------

% Recuo do parágrafo :
\setlength{\parindent}{1.5cm}
\setlength{\parskip}{0cm} 
\geometry{a4paper}%,left=3cm,right=2cm,top=3cm,bottom=2cm} % SEMPRE BOM VER A DOCUMENTAÇÃO
\hoffset = -1in
\voffset = -1in
\oddsidemargin = 1.5cm
\topmargin = 2cm
\headheight = 0pt
\headsep = 6pt % multiplicar por 0.03515 para obter cm
\textheight = 249mm
\textwidth = 18cm
\marginparsep = 0pt
\marginparwidth = 0pt
\footskip = 6mm
%
%
%
\usepackage{array}
\newcolumntype{L}[1]{>{\raggedright\let\newline\\\arraybackslash\hspace{0pt}}m{#1}}
\newcolumntype{C}[1]{>{\centering\let\newline\\\arraybackslash\hspace{0pt}}m{#1}}
\newcolumntype{R}[1]{>{\raggedleft\let\newline\\\arraybackslash\hspace{0pt}}m{#1}}
%
%
%
\newtheorem{teorema}{Teorema}[section]
\newtheorem{propriedade}[teorema]{Condição}
\newtheorem{lema}[teorema]{Lema}
\newtheorem{corolario}[teorema]{Corolário}
\newtheorem{proposicao}[teorema]{Proposição}
\newtheorem{definicao}[teorema]{Definição}
\newtheorem{exemplo}[teorema]{Exemplo}
\newtheorem{af}[teorema]{AF}
\newtheorem{exercicio}[teorema]{Exercício}
\newtheorem{algoritmo}{Algoritmo}[section]
\newtheorem{exc}{}[section]
\newtheorem{impl}{}[section]
\newtheorem{hip}{H\hspace{-.12cm}}
\newtheorem{res}{R\hspace{-.12cm}}
%
%
%
\newcommand{\rel}{\Bbb R^{\ell}}
\newcommand{\rell}{\mathbb R^{\ell}_{+}}
\newcommand{\rr}{\mathbb R}
\newcommand{\bb}[1]{\boldsymbol{#1}}
\newcommand{\ol}[1]{\overline{#1}}
\newcommand{\hess}{\nabla^2(f(\bb{x}))}

\newcommand{\kktmax}[3]{
\begin{aligned}
& \underset{#1}{\text{maximizar}}
& & {#2} \\
& \text{sujeito a} 
& & {#3} \\
\end{aligned}}

\newcommand{\kktmin}[3]{
\begin{aligned}
& \underset{#1}{\text{minimizar}}
& & {#2} \\
& \text{sujeito a} 
& & {#3} \\
\end{aligned}}

%  Novos macros
\newcommand{\R}{\mathbb{R}}
\newcommand{\N}{\mathbb{N}}
\newcommand{\Rn}{{\R}^n}
\newcommand{\sumin}{\sum_{i=1}^n}
\newcommand{\Rl}{{\R}^\ell}
\newcommand{\Rm}{{\R}^m}
\newcommand{\ind}{1,\ldots,n}
%
%
%
%
%


\title{RIGIDEZ AMBIENTAL E COMÉRCIO INTERNACIONAL: O QUE EVIDENCIAM OS DADOS?}

\author{Pedro Milreu Cunha\thanks{Mestrando em economia aplicada no Programa de Pós-Graduação em Economia, Universidade Federal da Paraíba (UFPB). Email: <pedro.milreu@academico.ufpb.br>.} \and Rafael de Sousa Araújo\thanks{Doutorando em economia no Programa de Pós-Graduação em Economia, Universidade Federal de Pernambuco (UFPE). Email: <rafaelaraujo05@hotmail.com>.} \and Viviani Silva Lírio\thanks{Professora Titular do Departamento de Economia Rural da Universidade Federal de Viçosa (UFV). E-mail: <vslirio@ufv.br>.}}

\date{}
%
%
%
%
%
%
%
%
%
